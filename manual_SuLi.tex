\documentclass[12pt, a4paper]{article}

\usepackage[pdftex]{color,graphicx}
\usepackage{appendix}
\usepackage{indentfirst}
\usepackage{verbatim}
\usepackage{listings} % http://ctan.org/pkg/listings % http://linorg.usp.br/CTAN/macros/latex/contrib/listings/listings-devel.pdf
\lstset{
  basicstyle=\ttfamily,
  mathescape=True,
  escapeinside='',
  breaklines=true
}

\newcommand{\red}{\textcolor{red}}
\newcommand{\SL}{{\bf SuLi}}
\newcommand{\SLIT}{{\bf SuLi-IT}}
\newcommand{\SLIC}{{\bf SuLi-IC}}

\usepackage{amsmath, amsfonts, amssymb}
\usepackage{graphicx}
%\usepackage{listings}
%\lstset{language=C}
\usepackage{comment}
%\usepackage[dvips]{hyperref}
\usepackage[colorlinks=true,linkcolor=blue,filecolor=blue,urlcolor=blue,citecolor=blue]{hyperref}
\usepackage{psfrag}
\usepackage{mathtools}
\usepackage{pgfplots}
\pgfplotsset{/pgf/number format/use comma,compat=newest}

\usepackage[T1]{fontenc}
\usepackage[utf8]{inputenc}
\usepackage[portuguese]{babel} % English language/hyphenation
%\usepackage[Brazil]{babel}

\usepackage[natbibapa]{apacite}% Citation support using apacite.sty. Commands using natbib.sty MUST be deactivated first!

%\hypersetup{
%colorlinks=true,
%linkcolor=blue,           
%}

\makeatletter
\def\@xobeysp{ }
\makeatother

\usepackage[top=1.5cm, bottom=2cm, left=2cm, right=2cm]{geometry}

\newcommand{\p}{\partial}
\newcommand{\bomega}{\boldsymbol{\omega}}
\newcommand{\bnabla}{\boldsymbol{\nabla}}

% Useful packages
\usepackage{multirow}
\usepackage{afterpage}
\usepackage{hyperref}
\usepackage{xcolor}
\graphicspath{{images/}}
\usepackage{listings}
\usepackage[nottoc]{tocbibind}
\usepackage{gensymb}
\usepackage{tikz,tkz-tab}
\usepackage{lastpage}

% Meta content 
\title{Manual do Usuário}
\author{
  Leonardo Romero Monteiro \\[.5 cm]
}
\date{Setembro de 2024}
\pagestyle{headings}
\makeatletter
\let\thetitle\@title
\let\theauthor\@author
\let\thedate\@date
\makeatother

%Color definitions
\definecolor{dark}{RGB}{22,75,94}
\definecolor{lightgrey}{rgb}{.90,.90,.90}
\definecolor{codegreen}{rgb}{0,0.6,0}
\definecolor{codegray}{rgb}{0.5,0.5,0.5}
\definecolor{codepurple}{rgb}{0.58,0,0.82}
\definecolor{backcolour}{rgb}{0.95,0.95,0.92}
 
\lstdefinestyle{mystyle}{
    backgroundcolor=\color{backcolour},   
    commentstyle=\color{codegreen},
    keywordstyle=\color{magenta},
    numberstyle=\tiny\color{codegray},
    stringstyle=\color{codepurple},
    basicstyle=\footnotesize,
    breakatwhitespace=false,         
    breaklines=true,                 
    captionpos=b,                    
    keepspaces=true,                                    
    numbersep=5pt,                  
    showspaces=false,                
    showstringspaces=false,
    showtabs=false,                  
    tabsize=2
}
\lstset{style=mystyle}

\hypersetup{
    colorlinks=true,
    linkcolor=black,
    filecolor=magenta,     
    urlcolor=blue,
    citecolor=blue,
}

%%%%%%%%%%%%%%%%%%%%%%%%%%%%%%%%%%%%%%%%%%%%%%%%%%%%%%%%%%%%%%%%%%%%%%%%%%%%%%%%%%%%%%%%%%%%%%

\begin{document}

\begin{titlepage}
$ $

\vspace{3cm}
\begin{center}

\begin{figure}[htp]
\centering
\includegraphics[width=0.25\textwidth]{suli}
\end{figure}

\vspace{1cm}

{\LARGE SuLi - Simulador de Escoamentos com Superfície Livre}

	\noindent\rule{\linewidth}{0.2 mm} \\[1 cm]
	{\LARGE \bfseries \thetitle}\\[.7 cm]
	\rule{\linewidth}{0.2 mm} \\[1 cm]

	{ \Large \theauthor}
	\end{center}

\vspace{6cm}

	\begin{flushleft}
	  Versão 2.0 \\  \thedate
	\end{flushleft}
	
\end{titlepage}

\newpage

\begin{titlepage}

\begin{table}[htp]
\centering
\begin{tabular}{ccc}
\multirow{4}{*}{\includegraphics[width=0.15\textwidth]{udesc}} & Universidade do Estado de Santa Catarina & \multirow{4}{*}{\includegraphics[width=0.15\textwidth]{lacia}} \\
                  & Centro de Ciências Tecnológicas & \\
                  & Departamento de Engenharia Civil & \\
                  & Laboratório de Ciências das Águas &                  
\end{tabular}
\end{table}

\centering \& \\[.5 cm]  

\begin{table}[htp]
\centering
\begin{tabular}{ccc}
\multirow{3}{*}{\includegraphics[width=0.15\textwidth]{ufrgs}} & Universidade Federeal do Rio Grande do Sul & \multirow{3}{*}{\includegraphics[width=0.10\textwidth]{iph}} \\
                  & Instituto de Pesquisas Hidráulicas & \\
                  & Núcleo de Estudos em Transição e Turbulência & \\               
\end{tabular}
\end{table}

\vspace{3cm}

\begin{flushleft}
	
\noindent\rule{\linewidth}{0.2 mm} \\[1 cm]
	
Monteiro, Leonardo Romero \\[1 cm]
	
Manual do Usuário: SuLi - Simulador de Escoamentos com Superfície Livre - Joinville/SC: UDESC/CCT/DEC/LaCiA, 2024. \\[0.2 cm]
	
\pageref{LastPage} Páginas \\[1 cm]
	
Palavras-chave:

Equações de Navier-Stokes, Escoamento monofásico, Escoamento bifásico, Simulador Numérico. 
	
\rule{\linewidth}{0.2 mm} \\[1 cm]
	
Colaboradores:

Bruna Fernanda Soares (2022-2024) \\
Karol Rocha Araújo (2023-2024) \\
Pedro Miguel de Andrade Filho (2022-2023) \\
Vinícios Morigi (2021-2022)
 
\vspace{3cm}

Versão 2.0 \\ Edição brasileira \\ \thedate
\end{flushleft}
	
\end{titlepage}

\newpage

\begin{titlepage}

\tableofcontents 

%\listoffigures \newpage CASO NECESSÁRIO!

%\listoftables \newpage CASO NECESSÁRIO!

\end{titlepage}

\newpage

%%%%%%%%%%%%%%%%%%%%%%%%%%%%%%%%%%%%%%%%%%%%%%%%%%%%%%%%%%%%%%%%%%%%%%%%%%%%%%%%%%%%%%%%%%%%%%
%%%%%%%%%%%%%%%%%%%%%%%%%%%%%%%%%%%%%%%%%%%%%%%%%%%%%%%%%%%%%%%%%%%%%%%%%%%%%%%%%%%%%%%%%%%%%%

\section{Introdução}
O programa {\bf SuLi (Superfície Livre)} foi criado em 2014 pelo Núcleo de Estudos em Transição e Turbulência (NETT) pertencente ao Instituto de Pesquisas Hidráulicas da Universidade Federal do Rio Grande do Sul (IPH/UFRGS). Atualmente o programa é também desenvolvido pelo Laboratório de Ciência das Águas (LaCiA) no Centro de Ciências Tecnológicas da Universidade do Estado de Santa Catarina (CCT/UDESC).

O código possui dois módulos para representação da superfície livre numericamente. Um denominado \SLIT\ que utiliza métodos de rastreamento de interface (\textit{interface-tracking method}) o qual simula o fluido de forma monofásica (\cite{monteiro2014}) e outro \SLIC\ que utiliza o método de captura de interface (\textit{interface-capturing method}), baseado na metodologia Level-Set (\cite{monteiro2019}) o qual considera uma composição bifásica do escoamento. Também é possível considerar um contorno sólido variável aplicando a estratégia do Método de Fronteiras Imersas (\textit{Immersed Boundary Method} - IBM) por forçagem direta (\textit{discrete forcing}). As etapas de cálculo de desenvolvimento do modelo matemático são apresentadas através do fluxograma na Figura \ref{fig:suli}. %\label{itxic}

    \begin{figure} [!htb] 
    \centering
    \caption{Fluxograma de desenvolvimento do modelo matemático do código SuLi.}
    \includegraphics[width=13cm]{Suli_fluxograma.jpeg}
    %\fonte{Elaborado pelos autores (2024).}
    \label{fig:suli}
    \end{figure}

O código é voltado para o desenvolvimento acadêmico e quando aplicado em casos reais de engenharia, o usuário deve conhecer especificamente quais métodos são os mais adequados para cada uso e aplicação. Os desenvolvedores do código não se responsabilizam perante os resultados gerados.

\subsection{Objetivo}
O \SL\ tem como objetivo ser um código aberto e livre que simula numericamente corpos de água e escoamentos com superfície livre utilizando as equações de Navier-Stokes, fornecendo resultados capazes de suprir análises importantes destes escoamentos. O usuário tem a liberdade de definir as condições iniciais e de contorno desejadas, além de modificar o código como assim desejar para atingir os seus objetivos.

\subsection{Considerações adicionais}
O código é escrito na linguagem de programação Fortran 90 desenvolvido na plataforma Linux, utilizando o programa Gfortran (ao usuário do Windows é possível utilizar o código após algumas configurações também abordadas neste manual, vide seção \ref{WindowsUsers}).

Para o pós-processamento dos resultados é recomendado o programa livre \verb|Paraview| (abordado na seção \ref{visu}), entretanto o usuário tem a liberdade de utilizar qualquer outro programa de seu interesse, bastando adaptar o formato dos dados de saída do \SL\ de acordo com o programa escolhido.

Ao utilizar o módulo \SLIT\ favor referenciar a publicação \cite{monteiro2014} e caso utilize \SLIC\ favor referenciar a publicação \cite{monteiro2019}. O \SL\ é um código livre e aberto e a única forma de fortalecer o mesmo é ampliando a sua utilização.

\newpage

%%%%%%%%%%%%%%%%%%%%%%%%%%%%%%%%%%%%%%%%%%%%%%%%%%%%%%%%%%%%%%%%%%%%%%%%%%%%%%%%%%%%%%%%%%%%%%
%%%%%%%%%%%%%%%%%%%%%%%%%%%%%%%%%%%%%%%%%%%%%%%%%%%%%%%%%%%%%%%%%%%%%%%%%%%%%%%%%%%%%%%%%%%%%%

\section{O SuLi}
Para poder utilizar o \SL\ da melhor forma, é necessário que alguns pré-requisitos de funcionamento sejam instalados e configurados em sua máquina, em contra partida não é essencial o conhecimento avançado do código para o seu uso mais básico, basta um entendimento geral conforme aborda a seção \ref{utilizando_o_codigo}.

\subsection{Pré-requisitos}
Os requisitos necessários para a melhor utilização do \SL\ solicitam um computador com sistema operacional 64-bit. A configuração do processador e da memória RAM dependerão dos casos que se deseja simular, mas, de qualquer forma, são indicados computadores com no mínimo: 4 GB de RAM, e; processador i5 ou equivalente. Aconselha-se também se ter ao menos 5 GB livres no HD antes de iniciar qualquer simulação. Quanto melhor for o computador mais rápidas serão as simulações e mais pontos podem ser utilizados no domínio computacional. 

\paragraph{Observação:} Os pré-requisitos são apenas necessários para usuários iniciantes. Usuários avançados podem adaptar as configurações de suas máquinas de acordo com suas necessidades.

\subsection{Instalação}
\subsubsection{Usuários de Windows}	\label{WindowsUsers}
Para usuários do Windows, há uma etapa de instalação inicial extra, que envolve a configuração do ambiente Ubuntu. É necessário que siga os passos contidos no Apêndice \ref{InstallUbuntu} e depois retorne a este tópico, prosseguindo a instalação naturalmente idealizada para usuários Linux.

%Update e Upgrade do sistema%
\subsubsection{Para usuários Linux e Windows}  \label{ContinueInstall}
Para usuários Linux (ou Windows que já cumpriram o passo anterior), é necessário garantir que o sistema operacional está atualizado a fim de evitar erros de compatibilidade. Para tal, siga os passos seguintes no terminal de comando do Linux:

\begin{verbatim}
sudo apt-get update (Para compreender os programas e arquivos desatualizados)
\end{verbatim}

\begin{verbatim}
sudo apt-get upgrade (Para efetivamente atualizar o sistema)
\end{verbatim}

%gfortran e Makefile%
Prosseguindo, é necessário instalar os programas {\bf gfortran}, {\bf MakeFile} e {\bf Biblioteca FFT} em sua máquina, para que o código do \SL\ possa ser corretamente interpretado e compilado. Portanto, ainda no terminal, siga estes comandos para realizar a instalação:

\begin{verbatim}
sudo apt-get install gfortran	(Para instalar o gfortran)
\end{verbatim}

\begin{verbatim}
sudo apt-get install make	(Para instalar o MakeFile)
\end{verbatim}

\begin{verbatim}
sudo apt-get install libfftw3-dev	(Para instalar a biblioteca FFT)
\end{verbatim}

Agora está tudo instalado e configurado para a utilização do \SL. A instalação do visualizador de dados Paraview é apresentada na seção \ref{visu}. Caso a instalação de algum desses pacotes não seja efetuada com sucesso, tente refazer as etapas de configuração, instalação e atualização do sistema. Caso o problema persista entre em contato pelo e-mail: leonardo.monteiro@udesc.br

\subsection{Estrutura}
Para a utilização do \SL\ existem 3 etapas:
\begin{itemize}
	\item Compilação e inicialização do programa:
		\begin{itemize}
			\item Arquivo gerador/compilador inicial ``Makefile''.
		\end{itemize}
	\item Cálculo da simulação numérica:
		\begin{itemize}
			\item Algoritmos em Fortran 90 programados para efetivamente realizar a simulação. As condições iniciais e de contorno estão alocadas nesta etapa.
		\end{itemize}
	\item Armazenamento dos dados obtidos:
		\begin{itemize}
			\item Pastas ``dados'' e ``arquivos'' que armazenam os dados para etapas de pós-processamento e análises futuras.
		\end{itemize}
\end{itemize}

Na seção \ref{utilizando_o_codigo} a apresentação do código é feita de forma detalhada, fornecendo ferramentas para que o usuário compreenda a liberdade e tenha autonomia de personalizar o programa de acordo com a sua necessidade.

\subsection{Compilando}
A partir do terminal de comando do Linux, navegue até o local onde está a pasta do \SL . Ciente dos arquivos presentes na pasta, o usuário pode realizar algumas operações no terminal de fomar individual com o {\bf{Makefile}}, que são:
\label{limpar_dados}
\begin{verbatim}
make          (para compilar o código)*; 
make clean    (para remover os arquivos auxiliares criados na compilação);
make allclean (para remover todos os arquivos não básicos para o código).
\end{verbatim} 

\paragraph{*}	\label{infosmake}
O comando {\bf{make}} fornece liberdade ao usuário para definir itens específicos relevantes à compilação dentro do arquivo Makefile. Ativando itens na linha ``OPTFC ='' mudará importantes configurações de compilação, por exemplo:

\begin{itemize}
    \item\textbf{-mcmodel=medium} $\rightarrow$ possibilitará o uso de mais pontos para a simulação. Utilizar se estiver ocorrendo erro por causa do excesso de pontos.

    \item\textbf{-fcheck=all} $\rightarrow$ faz uma verificação do código para identificar se está escrito corretamente. Utilizar quando se modificar bruscamente o código para identificar possíveis erros de escrita.
\end{itemize}

Usuários que não tenham conhecimento do uso do Makefile devem utilizar a opção padrão definida.

Toda vez que o usuário modificar algo na escrita das rotinas no código (nos arquivos .f90) deve ser feita novamente a compilação com o comando \verb!make!. Conduto, se a mudança for em um módulo, deve-se primeiro remover os arquivos auxiliares (\verb!make clean!) e depois realizar a compilação (\verb!make!).

\subsection{Rodando}
Depois de compilado, para rodar o \SL\ basta digitar o comando \verb!./rodar_3d! no terminal (note que tanto a pasta na qual o programa foi instalado quanto as pastas ``dados'' e ``arquivos'' vão receber novos arquivos, auxiliares ao processo da simulação). Para interromper a simulação, feche o terminal de comando ou aperte o comando \verb!Ctrl+c!.

%\subsection{Problemas conhecidos}

\newpage

%%%%%%%%%%%%%%%%%%%%%%%%%%%%%%%%%%%%%%%%%%%%%%%%%%%%%%%%%%%%%%%%%%%%%%%%%%%%%%%%%%%%%%%%%%%%%%
%%%%%%%%%%%%%%%%%%%%%%%%%%%%%%%%%%%%%%%%%%%%%%%%%%%%%%%%%%%%%%%%%%%%%%%%%%%%%%%%%%%%%%%%%%%%%%

\section{Personalizando o código}	\label{utilizando_o_codigo}

Nesta seção são definidos: domínio de cálculo, funções a serem utilizadas para o cálculo das equações, condições iniciais e condições de contorno. Incluem-se as principais modificações e personalizações que o usuário pode fazer de acordo com a sua necessidade.

Mesmo que o \SLIT\ e o \SLIC\ sejam semelhantes, existem funções/variáveis particulares de cada código. Quando o item for específico do \SLIT\ a função/variável terá sobrescrito $^{IT}$, caso for específico do \SLIC\ terá sobrescrito $^{IC}$.

O \SL\ é composto de códigos reunidos em documentos de subrotina, dos quais se complementam e se interdependem. Abaixo são as pastas que compõem o \SLIC\ .

\begin{table}[htb]
	\begin{center}
		\begin{tabular}{llll}
		0\_modules.f90 & 1\_principal.f90 & 2\_iniciais.f90 & 3\_contorno.f90 \\
		4\_plot.f90  & 5\_viscosidade.f90 & 6\_convdiff.f90 & 7\_advectivo.f90 \\
		8\_derivadas.f90 & 9\_grad.f90 & 10\_pos.f90 & 11\_diversos.f90 \\
		12\_paraview\_3D.f90 & 13\_mms.f90 & 14\_level\_set.f90² & 15\_ibm.f90
		\end{tabular}		
	\end{center}
\end{table}

\paragraph{Atenção!} A atribuição de parâmetros para uma simulação deve ser feita no documento \,\verb|00_suli3d.prm|. Feito isso, o parâmetro será automaticamente aplicado em sua respectiva pasta de subrotina.

\paragraph{Observação:} Sempre que uma nova simulação for realizada é importante limpar todos os dados existentes referentes a simulações anteriores para que estes não influenciem no andamento e resultado da nova simulação. Para tanto, utilize os comandos apresentados em \ref{limpar_dados} de acordo com a necessidade.

\subsection{Definindo o domínio}

Na pasta \,\verb|00_suli3d.prm| são encontradas as variáveis para especificar as dimensões espaciais pelo o número de células do domínio e discretização do intervalo espacial, \verb|(nx,ny,nz)| e \verb|(dx,dy,dz)|, respectivamente. As dimensões do domínio são referenciadas por:
\begin{lstlisting}
Eixo x  $\rightarrow$   $(nx+1) \times dx$  $\rightarrow$  Comprimento (m)
Eixo y  $\rightarrow$   $(ny+1) \times dy$  $\rightarrow$  Largura (m)
Eixo z	$\rightarrow$   $(nz+1) \times dz$  $\rightarrow$  Profundidade (m)
\end{lstlisting}	\label{eixos}

A malha temporal da simulação é definida pelas seguintes variáveis:
\begin{lstlisting}
dt0       $\rightarrow$  'Discretização' temporal $\Delta t$ (s)
dt_frame  $\rightarrow$  'Número' de tempo por arquivo plotado (s)
t_s       $\rightarrow$  Tempo a ser simulado (s)
\end{lstlisting}

\paragraph{Observação:} O domínio espacial do \SL\ foi feito para realizar simulações em três dimensões (3D) e não em apenas duas (2D). Para tanto, o usuário que quiser minimizar os efeitos de uma terceira dimensão em sua simulação deverá possuir um bom entendimento das equações e leis que regem tais processos, para que os resultados obtidos estejam condizentes com a realidade física e matemática proposta à simulação.

\newpage

\subsection{Condições e configurações iniciais}
Ainda no \verb|00_suli3d.prm|, a primeira condição inicial importante em relação ao escoamento é o parâmetro \verb|uinicial|; que significa a velocidade inicial que o escoamento terá a partir do seu \verb|eixo x|, isto é, ao longo do comprimento do seu escoamento (como denotado em \ref{eixos}). Também existem algumas outras configurações iniciais importantes para o desenvolvimento da simulação, neste e em outros arquivos/módulos/subrotinas, tratados a seguir:

\subsubsection{Evolução temporal²}
Diferentes métodos para calcular a evolução temporal da solução estão disponíveis no \SLIC. Os métodos podem resultar em um grau menor de precisão e com maior rapidez na resolução, ou serem mais precisos, porém consequentemente mais lentos. Esta função está disponível pelo parâmetro representado por \verb|t_tempo|, acessível em \verb|00_suli3d.prm| e aplicado em \verb|11_diversos.f90|.

\begin{lstlisting}[escapeinside='']
t_tempo	
  0  $\rightarrow$  Euler 'Explícito'
  1  $\rightarrow$  Runge-Kutta de '2ª Ordem'
  2  $\rightarrow$  Runge-Kutta de '3ª Ordem'
  3  $\rightarrow$  Adams-Bashforth de '2ª Ordem'
\end{lstlisting}

Para casos com tipo de tempo variável, pode ser definido em \verb|t_tempo_var|.

\begin{lstlisting}[escapeinside='']
t_tempo_var	
  0  $\rightarrow$  $\Delta t$ constante
  1  $\rightarrow$  $\Delta t$ adaptativo	
\end{lstlisting}

\paragraph{Euler Explícito de 1ª Ordem (EE):} Mais simples, mais rápido, menos preciso.
\paragraph{Runge-Kutta de 2ª Ordem (RK2):} Mais complexo, mais lento, mais preciso.
\paragraph{Runge-Kutta de 3ª Ordem (RK3):} Ainda mais complexo, ainda mais lento, ainda mais preciso.
\paragraph{Adams-Bashforth de 2ª Ordem (AB2):} Menor custo computacional se comparado ao método de RK2, em contrapartida possui restrições quanto a estabilidade do código.

\subsubsection{Termo advectivo}
É possível obter formas matemáticas distintas, porém equivalentes, para o cálculo dos termos advectivos da equação de Navier-Stokes. Mesmo sendo equivalentes matematicamente, podem apresentar diferenças numéricas relevantes. Para modificar a forma de cálculo do termo advectivo utilizamos o parâmetro representado por \verb|adv_type| aplicado no módulo \verb|7_advectivo.f90|. Para a modificação deste parâmetro, também deve-se acessar a pasta \verb|00_suli3d.prm|.

\begin{lstlisting}[escapeinside='']
adv_type  
  1  $\rightarrow$  Advectivo 'Clássico'
  2  $\rightarrow$  Rotacional
  3  $\rightarrow$  'Antissimétrico'
\end{lstlisting}

\paragraph{Formulação Clássica:} $H_i = u_j  \dfrac{\partial u_i}{\partial x_j}$
\paragraph{Formulação Rotacional:} $H_i = u_j \left(\dfrac{\partial u_i}{\partial x_j} - \dfrac{\partial u_j}{\partial x_i}\right)$
\paragraph{Formulação Antissimétrica:} $H_i = \dfrac{1}{2} \left(\dfrac{\partial u_i u_j}{\partial x_j} + u_j\dfrac{\partial u_i}{\partial x_j}\right)$

\subsubsection{Termo convectivo}
As derivadas espaciais do termo convectivo são resolvidas por aproximações numéricas das mesmas. Os tipo de derivada que o \SL\ utiliza para as resoluções dos métodos através de determinados esquemas de interpolação são escolhidos pelo parâmetro \verb|der|, antes determinadas em \verb|00_suli3d.prm|, e aplicadas no arquivo \verb|0_module.f90|.
\begin{lstlisting}[escapeinside='']
der	
  1  $\rightarrow$  Upwind
  2  $\rightarrow$  Centrado$^*$
  3  $\rightarrow$  Upwind de $2^a$ Ordem$^{IC}$
\end{lstlisting}

\paragraph{\textit{Upwind}:} Proposta para o cálculo de soluções aproximadas de equações de conservação. Possui grande erro de difusão numérica. Também chamado de \textit{Upwind} de 1ª Ordem.
\paragraph{Centrado$^{*}$:} Simples, preciso e rápido mas pode facilmente apresentar problemas de estabilidade numérica.
\paragraph{\textit{Upwind} de 2ª Ordem$^{*,IC}$:} Mais preciso do que o \textit{Upwind} de 1ª Ordem, entretanto mais custoso. Ainda possui erro expressivo de difusão numérica. 

\vspace{0.5cm}
\noindent *Só é possível utilizar as opções quando \verb|adv_type| for advectivo clássico.

\subsubsection{Obstáculos} \label{obst}
É possível adicionar obstáculos ao domínio da simulação do escoamento a partir da metodologia IBM (\textit{Immersed Boundary Method}), dado por \verb|ibm_t|. O IBM é utilizado para resolver situações onde há interação entre fluidos e objetos sólidos, como por exemplo, a simulação de escoamento ao redor de uma estrutura ou a movimentação de partículas dentro de um fluido.

\begin{lstlisting}[escapeinside='']
ibm_t	
  0  $\rightarrow$  Sem IBM e sem 'obstáculo'
  1  $\rightarrow$  IBM 'forçado'
  2  $\rightarrow$  IBM induzido - '\cite{auguste2019}'
\end{lstlisting}

Os obstáculos são representados no código através do parâmetro \verb|obst_t| (definido na pasta \verb|00_suli3d.prm|) e modificados no módulo \verb|3_contorno.f90|. No \SLIT\ estão disponíveis 8 obstáculos pré-definidos, enquanto no \SLIC\ constam 2, conforme segue:

\begin{lstlisting}[escapeinside='']
obst_t$^{IT}$	
  0  $\rightarrow$  Sem 'obstáculo'
  1  $\rightarrow$  Dunas
  2  $\rightarrow$  '\cite{yue2003}'
  3  $\rightarrow$  '\cite{nasr2014}'
  4  $\rightarrow$  '\cite{beji1994}'
  5  $\rightarrow$  Canal tipo 0 - '\cite{delft1980}'
  6  $\rightarrow$  Canal tipo 1 e 2 - '\cite{delft1980}'
  7  $\rightarrow$  '\cite{monteiro2017}'
  8  $\rightarrow$  '\cite{rigotti2015}'
obst_t$^{IC}$	
  0  $\rightarrow$  Sem 'obstáculo'
  1  $\rightarrow$  Rugosidade de fundo uniforme - '\cite{zampiron2022}'
  2  $\rightarrow$  Barragem - Koshizuka (1995) e Kleefsman (2005)
\end{lstlisting}

Assim como o usuário pode optar por não adicionar obstáculo ao seu domínio (\verb|obst_t = 0|), é possível que em versões futuras do \SL\ e deste manual sejam disponibilizadas ferramentas e opções para modelagem de obstáculos diretamente pelo próprio usuário.

\paragraph{Sem IBM:} O IBM está desativado e não há obstáculos no domínio.
\paragraph{IBM forçado:} Utilizado para forçar a velocidade a zero nas células da malha que correspondem à região ocupada pelo corpo sólido dentro do fluido; significa que o movimento do fluido é impedido dentro dessas células, criando um "bloqueio" que simula o efeito de um obstáculo rígido.
\paragraph{IBM induzido:} As forças de interação entre o fluido e o obstáculo são calculadas de maneira a induzir o comportamento desejado no fluido ao redor do obstáculo. Esta técnica é baseada no trabalho de \cite{auguste2019}.

\subsubsection{Modelo de turbulência}
Utilizado para calcular e antecipar os efeitos da turbulência em fluidos, através da resolução numérica das equações de Navier-Stokes. O usuário pode escolher entre a não aplicação de um modelo de turbulência ou entre os diferentes modelos disponíveis quando do uso de uma malha não tão refinada. O uso de cada opção depende do nível de refino da malha utilizada na simulação, sendo a viscosidade total obtida pelo somatório da viscosidade cinemática com a viscosidade cinemática turbulenta. Este parâmetro é representado por \verb|m_turb|, e definido também no \verb|00_suli3d.prm|.

\begin{lstlisting}[escapeinside='']
m_turb  
  0  $\rightarrow$  Sem Modelo
  1  $\rightarrow$  LES Smagorinsky-Lilly 'Clássico' - '\cite{smagorinsky1963, lilly1966}'
  2  $\rightarrow$  RANS com energia 'cinemática' turbulenta - '\cite{heinz2020}'
  3  $\rightarrow$  DES - '\cite{heinz2020}'
\end{lstlisting}

\paragraph{Sem modelo:} Quando a malha estiver refinada o suficiente, excluindo a necessidade do uso de um modelo de turbulência ou outros métodos estejam sendo utilizados para representar a turbulência.
\paragraph{LES Smagorinsky-Lilly Clássico:} Modelo de simulação de grande escala (\textit{Large Eddy Simulation} - LES) que calcula a viscosidade cinemática turbulenta a partir do coeficiente de Smagorinsky (\verb|csmag|), do tamanho representativo da malha e do tensor de taxa de deformação.
\paragraph{RANS com energia cinemática turbulenta:} Modelo de Navier-Stokes por médias de Reynolds (\textit{Reynolds Average Navier Stokes} - RANS) que calcula a viscosidade cinemática turbulenta ($\nu_{t}$) a partir do parâmetro do modelo (\verb|cmu|), da energia cinética turbulenta e da escala de tempo de dissipação.
\paragraph{\textit{Detached Eddy Simulation} (DES):} Modelo híbrido RANS-LES que calcula a viscosidade cinemática turbulenta a partir dos parâmetros do modelo (\verb|cmu| e \verb|cdes|), da energia cinética turbulenta e da escala de tempo de dissipação. 

\paragraph{} Tendo definido o modelo de turbulência, é necessário a aplicação de coeficiente a eles, definidos por \verb|cka|, \verb|csmag|, \verb|cmu|, \verb|cdes|. 

\begin{lstlisting}[escapeinside=''] 
cka    $\rightarrow$  Constante de 'Kármán' (usualmente 0,41)
csmag  $\rightarrow$  Coeficiente de Smagorinsky
cmu	   $\rightarrow$  Coeficiente de modelo de 'turbulência' LES/RANS
cdes   $\rightarrow$  Coeficiente de modelo de 'turbulência' DES (usualmente 0,65)
\end{lstlisting}

\paragraph{Observação 1:}  Recomenda-se a calibração dos coeficientes dos modelos de turbulência. 

\paragraph{Observação 2:}  Considerando a possibilidade de aplicação de uma velocidade de entrada (\verb|uinicial|), a energia cinética turbulenta inicial é considerada a partir da intensidade de turbulência inicial (\verb|iturb|) definida no \verb|00_suli3d.prm|.

\paragraph{Observação 3:} O comprimento de escala do modelo deve ser corrigido no denominador do calculo da taxa de dissipação do código \verb|5_viscosidade.f90|.

\subsubsection{Ondas de entrada}
O \SL\ tem a capacidade de simular a entrada de ondas no domínio espacial do escoamento, através da aplicação numérica da Teoria de Ondas de Stokes. Esta codificação foi baseada no trabalho de \cite{higuera2013}. Os parâmetro a serem representados por \verb|wave_t|, e também definidos na pasta \verb|00_suli3d.prm|, são:

\begin{lstlisting}[escapeinside='']
wave_t	
  0  $\rightarrow$  Sem onda
  1  $\rightarrow$  Onda de Stokes I
  2  $\rightarrow$  Onda de Stokes II
  5  $\rightarrow$  Onda de Stokes V
\end{lstlisting}

\paragraph{Sem onda:} ausência de ondas na simulação.
\paragraph{Onda de Stokes I:} ondas de menor amplitude e maior comprimento.
\paragraph{Onda de Stokes II:} ondas de amplitude e comprimento intermediários.
\paragraph{Onda de Stokes V:} ondas de maior amplitude e menor comprimento.

\subsubsection{Plotagem$^{IC}$}
Refere-se ao que será apresentado ao final dos cálculos realizados pelo \SL\ na simulação, sendo para a visualização/análise do usuário. Parâmetro aqui representado por \verb|t_plot|.

\begin{lstlisting}
t_plot	
  0  $\rightarrow$  Modo simples
  1  $\rightarrow$  Modo completo
\end{lstlisting}

Na seção \ref{visu} é possível entender melhor o que e como serão visualizados os itens da simulação.

\paragraph{Modo Simples:} Consideram-se os parâmetros \verb|Velocidade| do escoamento; \verb|Level-Set|, ferramenta utilizada pelo \SL\ para diferenciar os fluidos do domínio, em geral água e ar; e \verb|IBM| (\textit{Immersed Boundary Method}), uma técnica utilizada para os cálculos em simulações de interação fluido-estrutura.
\paragraph{Modo Completo:} Além dos anteriores, também são levados em consideração os parâmetros \verb|Pressão,  Vorticidade e Viscosidade| do escoamento.

\subsubsection{Método da Solução Manufaturada}
O Método da Solução Manufaturada (\textit{Method of Manufactured Solutions} - MMS) é utilizado para a verificação das aplicações numéricas, atribuído apenas aos desenvolvedores de esquemas numéricos a fim de viabilizar o seu uso no código. Este parâmetro é representado por \verb|mms_t| e se define em \verb|00_suli3d.prm|. A presente solução manufaturada é descrita com detalhes por \cite{wang2009}.

\begin{lstlisting}[escapeinside='']
mms_t	
  0  $\rightarrow$  Sem MMS
  1  $\rightarrow$  MMS permanente
  2  $\rightarrow$  MMS 'não' permanente
\end{lstlisting}

% Os parâmetros \textit{a} e \textit{h0}, encontradas no arquivo \verb|0_module.f90| referem-se a altura da onda e a profundidade do escoamento, respectivamente, para a verificação do código.

Mais informações a respeito de formulações e aplicações podem ser encontradas no artigo \cite{monteiro2019}.

\subsubsection{Coeficientes Específicos}
Existem ainda 4 parâmetros muito importantes para a simulação (aqui tratados como coeficientes), são eles: \verb|tetah|$^{IT}$,  \verb|chezy|, \verb|decliv|, e \verb|z0|$^{IC}$ encontrados em \verb|0_module.f90|, mas também determinados inicialmente em \verb|00_suli3d.prm|.

\begin{lstlisting}[escapeinside=''] 
tetah$^{IT}$  $\rightarrow$  Coeficiente ponderador de implicitude do termo da 'pressão'; aconselha-se manter o valor de 0,6 - '\cite{patnaik1987}'
chezy    $\rightarrow$  Coeficiente de Chezy (C); pode ser aproximado ao coeficiente de Manning ($n$) 'através' da 'equação' $C = Rh^{1/6} / n$
decliv   $\rightarrow$  Declividade (m/m) aplicada ao longo do comprimento do 'domínio'
z0$^{IC}$     $\rightarrow$  Altura da rugosidade de fundo (m)
\end{lstlisting}

% \paragraph{Ponderador $\theta^{IT}$:} Este é um coeficiente ponderador de implicitude do termo da pressão. Tem função apenas na versão monofásica Aconselha-se deixar sempre no valor de 0,6. Este parâmetro é baseado no trabalho de \cite{patnaik1987}.

% \paragraph{} Algumas outras condições iniciais devem ser impostas no arquivo \verb|2_iniciais.f90|. Variáveis como a velocidade ($u,v,w$), as pressões ($prd1$) e desnível¹ ($eta1$)  podem ser inseridas de acordo com o usuário sem grandes dificuldades (sempre respeitadas as condições de estabilidade do código).

% No \SLIC, o desnível da água (a posição da interface entre a água e o ar) deve ser definida na subtirona \verb|level_set_ini| dentro do arquivo \verb|14_level_set.f90|, melhor detalhado no subítem \ref{sub:levelset}.


\subsubsection{Camada Esponja}

A camada esponja é um artificio utizado para eliminar ondas de dentro do domínio computacional, muito utilizado como condição de contorno de saída para que as ondas não retornem quando passarem por esta condição de contorno. Na pasta \verb|00_suli3d.prm|, define-se:

\begin{lstlisting}[escapeinside='']
esp_type
  0  $\rightarrow$  Sem camada esponja
  1  $\rightarrow$  Considerar a profundidade
  2  $\rightarrow$  Desconsiderar a profundidade
  3  $\rightarrow$  'Método' da Tangente 'Hiperbólica'
\end{lstlisting}	

Mais características como o comprimento da camada espoja e sua posição podem ser alteradas da subtorina \textit{sponge\_layer}, dentro do arquivo \verb|3_contorno.f90|.


% \subsubsection{Ativação do coeficiente de Chezy¹}
% Está é uma simples função para ativar ou desativar o coeficiente de Chezy no contorno do domínio computacional.
	
% \begin{lstlisting}[escapeinside='']
% chezy_t	
%   $\rightarrow$  0 para Desconsiderar atrito no fundo
%   $\rightarrow$  1 para Considerar atrito no fundo
% \end{lstlisting}

 \subsubsection{Pressão Hidrostática}
O coeficiente \verb|t_press|, determinado no \verb|00_suli3d.prm|, especifica o tipo de cálculo de pressão a ser utilizado na simulação.

 \begin{lstlisting}[escapeinside='']
t_press
  0  $\rightarrow$  'Aproximção' 'Hidrostática' de 'pressão'
  1  $\rightarrow$  'Aproximação' 'Não-Hidrostática'
\end{lstlisting}

\paragraph{Aproximação Hidrostática:} Considera que a pressão no fluido varia apenas devido ao peso do próprio fluido
\paragraph{Aproximação Não-Hidrostática:} Considera todas as componentes da pressão, incluindo as variações dinâmicas causadas pelo movimento do fluido.É uma abordagem mais completa e precisa para simulações onde a pressão varia significativamente devido às forças inerciais e viscosas.

\subsubsection{Level-Set$^{IC}$} 
O \textit{Level-Set} é um método para caracterização de diferentes fluidos imiscíveis envolvidos na simulação (água e ar, por exemplo). No modulo \verb|00_suli3d.prm| é possível personalizar estas configurações, que serão aplicadas ao \verb|14_level_set.f90|, a partir dos seguintes parâmetros:

\paragraph{$\alpha_1$:} Número de células que farão parte da espessura da interface entre fluidos, representado por \verb|alpha1|.
\paragraph{$\rho$:} Massa específica ($kg/m^3$) dos fluidos 1 e 2, representadas por \verb|rho_f1| e \verb|rho_f2|, respectivamente.
\paragraph{$\mu$:} Viscosidade dinâmica ($Pa.s$) dos fluidos 1 e 2, representadas por \verb|mi_f1| e \verb|mi_f2|, respectivamente.
\paragraph{$\sigma$:} Tensão superficial($N/m$), representada por \verb|sigma|; pode ser ativada utilizando o comando \verb|t_tens = 1| ou desativada com \verb|t_tens = 0|.
\vspace{0.5cm}

Para isso, existem pré-definidas 7 possibilidades para a definição da interface entre o ar e a água onde:

\vspace{0.5cm}
\begin{lstlisting}[escapeinside='']
tipo
  1  $\rightarrow$  Onda 
  2  $\rightarrow$  Onda 2 ('Criação' de onda no 'início do domínio' baseado em '\cite{higuera2013}')
  3  $\rightarrow$  Barragem 
  4  $\rightarrow$  Gota 
  5  $\rightarrow$  '\cite{fennema1990}' ou '\cite{aureli2008}'
  6  $\rightarrow$  MMS - '\cite{wang2009}'
  7  $\rightarrow$  '\cite{koshizuka1995}'.
\end{lstlisting}

\paragraph{} Dessa forma, após definir o \verb|tipo| de interface, é possível configurar dados complementares para a simulação, como:
\paragraph{$\Lambda_x$:} Lâmbda X $(m)$, define o comprimento de uma onda ou perturbação ao longo do eixo x.
\paragraph{$\Lambda_y$:} Lâmbda X $(m)$, define o comprimento de uma onda ou perturbação ao longo do eixo y.
\paragraph{Amplitude $(m)$:} Refere-se à medida da variação máxima da altura do fluido causada por uma onda (\verb|tipo = 1|); Fator de adição do comprimento em todas as direções (\verb|tipo = 3|); Raio ou metade de uma aresta de uma gota (\verb|tipo = 4|).
\paragraph{Profundidade $(m)$:} Distância vertical do fundo do domínio até a superfície livre do fluido. Podendo ser: a profundidade do escoamento sem a onda em \verb|tipo = 1| ou comprimento da onda na direção z na simulação \verb|tipo = 3|.
\paragraph{Curvatura do Chanfro $(adimensional)$:} Curvatura do chanfro entre planos na barragem (\verb|tipo = 3|) e na gota (\verb|tipo = 4|).


\subsubsection{Restart$^{IC}$}
Esta função serve para auxiliar o usuário em situações em que exista a necessidade de interromper a simulação sem perder o progresso da mesma, fazendo um backup que possa ser utilizado para reiniciar a simulação. Os parâmetros representados por \verb|interv_rest e irest| são acessíveis em \verb|0_modules.f90|, módulo \textit{restart} e aplicados nas subrotinas \textit{restart\_ini} do arquivo \verb|11_diversos.f90|.
\begin{lstlisting}[escapeinside='']
interv_rest = intervalo de tempo a ser salvo para o restart
irest	
  0  $\rightarrow$  'simulação não é' um restart de outra
  1  $\rightarrow$  'simulação é' um restart de outra
\end{lstlisting}

%%%%%%%%%%%%%%%%%%%%%%%%%%%%%%%%%%%%%%%%%%

\subsection{Condições de contorno}
Em \verb|modules.f90|, devido ao seu domínio tridimensional, o \SL\ possui 6 contornos espaciais a serem definidos com até 5 opções diferentes, explicados logo na sequência:
\begin{lstlisting}[escapeinside='']
ccx0  0  $\rightarrow$  'periódico'
      1  $\rightarrow$  free-slip
      2  $\rightarrow$  no-slip
      3  $\rightarrow$  prescrita

ccxf  0  $\rightarrow$  'periódico'
      1  $\rightarrow$  free-slip
      2  $\rightarrow$  no-slip
      3  $\rightarrow$  prescrita
      4  $\rightarrow$  'saída' livre
      
ccy0  0  $\rightarrow$  'periódico'
      1  $\rightarrow$  free-slip
      2  $\rightarrow$  no-slip
      3  $\rightarrow$  prescrita
      
ccyf  0  $\rightarrow$  'periódico'
      1  $\rightarrow$  free-slip
      2  $\rightarrow$  no-slip
      3  $\rightarrow$  prescrita
      
ccz0  1  $\rightarrow$  free-slip
      2  $\rightarrow$  no-slip
      3  $\rightarrow$  prescrita
      
cczf  1  $\rightarrow$  free-slip
      3  $\rightarrow$  prescrita
\end{lstlisting}

\begin{verbatim}
Onde: 	
		cc 	=  condição de contorno
		x, y, z  =  eixos do escoamento
		0, f  =  início/fim do eixo de escoamento
\end{verbatim}

\paragraph{Periódico:} Tudo que entra, sai do domínio e vice-versa, seguindo um regime de fluxo periódico. É importante citar que adotando a condição periódica no início (\verb|ccx0, ccy0 e/ou ccz0|) é necessário adotar também para o final (\verb|ccxf, ccyf e/ou cczf|)
\paragraph{\textit{Free-slip}:} Deslizamento livre, não há velocidade normal, apenas tangencial.
\paragraph{\textit{No-slip}:} Ausência de deslizamento, portanto, velocidade normal e tangencial nulas.
\paragraph{Prescrita:} Velocidade imposta pelo usuário, prescrita conforme desejar.
\paragraph{Saída livre:} Fornece liberdade de saída para o escoamento, mas necessita a adição de uma camada esponja para suavizá-lo ao final do domínio.


%%%%%%%%%%%%%%%%%%%%%%%%%%%%%%%%%%%%%%%%%%


% \subsection{Funções auxiliares}
% Nesta seção estão abordados tópicos relacionados a funções/rotinas auxiliares á aplicação do \SL, em áreas ainda mais aprofundadas do código. Portanto, adverte-se a necessidade de possuir um entendimento maior do propósito de cada aplicação antes de alterá-la em código.

% \subsubsection{Obstáculos} \label{met_num}
% Assim como o usuário pode optar por não adicionar nenhum obstáculo ao seu domínio (\verb|obst_t = 0|), é possível que em versões futuras do \SL\ e deste manual sejam disponibilizadas ferramentas e opções para modelagem de obstáculos diretamente pelo próprio usuário. A seguir estão as opções pré-definidas citadas na seção \ref{obst} e suas respectivas referências, aplicadas ao código no módulo \verb|3_contorno.f90|:

% \begin{lstlisting}[escapeinside='']
% obst_t	
%   0  $\rightarrow$   Sem 'Obstáculo'
% 	1  $\rightarrow$   Dunas
% 	2  $\rightarrow$   '\cite{yue2003}'
% 	3  $\rightarrow$   '\cite{nasr2014}'
% 	4  $\rightarrow$   '\cite{beji1994}'
% 	5  $\rightarrow$   canal 0 - '\cite{delft1980}'
% 	6  $\rightarrow$   canal 1 e 2 - '\cite{delft1980}'
% 	7  $\rightarrow$   '\cite{monteiro2017}'
% 	8  $\rightarrow$   '\cite{rigotti2015}'
% \end{lstlisting}

% \subsubsection{Level-Set²} \label{sub:levelset}
% O Level-Set é um método para caracterização de diferentes fluidos imiscíveis envolvidos na simulação (água e ar, por exemplo). No modulo \verb|14_level_set.f90| é possível personalizar estas configurações a partir dos seguintes parâmetros:

% \paragraph{$\alpha_1$:} Número de células que farão parte da espessura da interface entre fluidos.
% \paragraph{$\rho$:} Massa específica ($kg/m^3$) dos fluidos 1 e 2, representadas por \verb|rho_f1| e \verb|rho_f2|, respectivamente.
% \paragraph{$\mu$:} Viscosidade dinâmica ($Pa.s$) dos fluidos 1 e 2, representadas por \verb|mi_f1| e \verb|mi_f2|, respectivamente.
% \paragraph{$\sigma$:} Tensão superficial ($N/m$)
% \vspace{0.5cm}

% Existem pré-definidas 7 possibilidades para a definição da interface entre o ar e a água onde:

% \vspace{0.5cm}
% \begin{lstlisting}[escapeinside='']
%   1 $\rightarrow$ Onda 
% 	2 $\rightarrow$  Onda 2 ('Criação' de onda no 'início do domínio' baseado em '\cite{higuera2013}')
% 	3 $\rightarrow$  Barragem 
% 	4 $\rightarrow$  Gota 
% 	5 $\rightarrow$  '\cite{fennema1990}' ou '\cite{aureli2008}'
% 	6 $\rightarrow$  MMS - '\cite{wang2009}'
% 	7 $\rightarrow$  '\cite{koshizuka1995}'.
% \end{lstlisting}

% \subsubsection{Restart²}
% Esta função serve para auxiliar o usuário em situações em que exista a necessidade de interromper a simulação sem perder o progresso da mesma, fazendo um backup que possa ser utilizado para reiniciar a simulação. Os parâmetros representados por \verb|interv_rest e irest| são acessíveis em \verb|0_modules.f90|, módulo \textit{restart} e aplicados nas subrotinas \textit{restart\_ini} do arquivo \verb|11_diversos.f90|.
% \begin{lstlisting}[escapeinside='']
% interv_rest = intervalo de tempo a ser salvo para o restart
% irest	
%   $\rightarrow$  0 para indicar se a 'simulação não é' um restart de outra
% 	$\rightarrow$  1 para indicar se a 'simulação é' um restart de outra
% \end{lstlisting}

\newpage

%%%%%%%%%%%%%%%%%%%%%%%%%%%%%%%%%%%%%%%%%%%%%%%%%%%%%%%%%%%%%%%%%%%%%%%%%%%%%%%%%%%%%%%%%%%%%%
%%%%%%%%%%%%%%%%%%%%%%%%%%%%%%%%%%%%%%%%%%%%%%%%%%%%%%%%%%%%%%%%%%%%%%%%%%%%%%%%%%%%%%%%%%%%%%

\section{Visualização dos resultados}	\label{visu}
Como já mencionado, a opção dos autores para visualização dos resultados da simulação é o programa livre \textbf{Paraview}, disponível em \verb|www.paraview.org/|, com versões disponíveis para usuários de Linux e/ou Windows. No Linux (Ambiente Ubuntu) a instalação do paraview pode ser facilmente realizada por meio do comando 

\begin{verbatim}
sudo apt install paraview
\end{verbatim}

No Linux, para visualizar os resultados dos campos das variáveis calculadas com o \SL\ deve-se acessar a pasta "arquivos" \, via terminal e digitar \verb|paraview visu.xdmf|. No Windows, o comando direto via terminal não é possível, sendo assim, a opção é iniciar o Paraview em janela independente e abrir a pasta \verb|visu.xdmf| diretamente nele.

%Prof.: Este comando está em funcionamento para versões do ParaView, instaladas no Ubuntu, anteriores a $5.0$. Para versões mais atuais, o usuário deve utilizar o ParaView pelo programa binário fornecido no site do próprio programa. (não sei se isso é verdade)

%P/ Pedro: Não sei se isso é verdade! Temos que verificar.

O paraview acessará então os dados calculados pelo \SL\ e os apresentará de acordo com as ferramentas disponíveis na versão e interface do programa. O intervalo de criação dos campos (os frames da simulação) para esta posterior visualização é definido pelo parâmetro \verb|dt_frame| determinado em\verb|00_suli3d.prm|

Assim, o usuário, tendo um conhecimento básico do programa, pode usufruir dos resultados de sua simulação.

\newpage

%%%%%%%%%%%%%%%%%%%%%%%%%%%%%%%%%%%%%%%%%%%%%%%%%%%%%%%%%%%%%%%%%%%%%%%%%%%%%%%%%%%%%%%%%%%%%%
%%%%%%%%%%%%%%%%%%%%%%%%%%%%%%%%%%%%%%%%%%%%%%%%%%%%%%%%%%%%%%%%%%%%%%%%%%%%%%%%%%%%%%%%%%%%%%

% \section{Paralelizando o código} % Karol (2024) - necessário complementar

% \subsection{Metodologias}
% Explicar o Cube GUI (com o Score-P)
% Open MP

% \subsection{Aplicação (passo a passo)}
% Cube GUI 

% \paragraph{1)} Baixar e instalar o ScoreP< Acesso em https://www.vi-hps.org/projects/score-p> e o CubeGUI <Acesso em: https://www.scalasca.org/software/cube-4.x/download.html> 

% \paragraph{2)} No MakeFile do SuLi, escrever 
% $CC = scorep$ 

% \paragraph{3)} Ainda no MakeFile, verificar se após o $'rodar\_3d'$ e o $'\%.o'$ está escrito respectivamente: \\

% \noindent \$(CC) \$(FC) -O3 -o \$@ \$(OBJ) \$(OPTFC) \\
% \$(CC) \$(FC) \$(OPTFC) \$(OPTIONS) -c \$< 

% \paragraph{4)} Tendo isso, rodar o programa normalmente com:  

% \vspace{0.5cm}
% \noindent make \\
% $.$/rodar\_3d \\

% Essas configurações no MakeFile, farão com que após rodado o programa, seja criado uma pasta com nome $'scorep-...'$, onde será armazenado os dados em questão. 

% \paragraph{5)} Para executar a pasta, digitar no mesmo terminal  \\

% \noindent square ./scorep-... \\

% Isso fará com que abra a interface do Cube GUI, onde mostrará, em tempo e percentual, quais sub-rotinas estão demandando mais tempo na simulação total. Abrindo a sub-rotina, é possível verificar em qual linha de paralelização está demandando a maior parte do tempo, para, assim, otimizar. 

% Como estamos utilizando 4 núcleos para as simulações, o tempo aparecerá multiplicado por 4. 

%\subsection{Resultados esperados}


%%%%%%%%%%%%%%%%%%%%%%%%%%%%%%%%%%%%%%%%%%
%\newpage
%\section{Metodologia e Aplicações Numéricas}
%%%%%%%%%%%%%%%%%%%%%%%%%%%%%%%%%%%%%%%%%%


\newpage
\appendix
\section{Apêndice: Ambiente Ubuntu no Windows}
\label{InstallUbuntu}

\paragraph{Preparando o Windows:} É necessário habilitar o sistema para aceitar qualquer distribuição Linux que venha a ser utilizada, para tal siga o caminho abaixo.

\begin{verbatim}
No "Menu Iniciar" do seu Windows vá para o "Painel De Controle".
Então clique em "Programas".
Depois clique em "Ativar ou desativar recursos do Windows".
Na janela que será aberta habilite a opção "Subsistema do Windows para Linux"
\end{verbatim}
Após ter realizado esse procedimento, é preciso reiniciar seu computador para que as alterações sejam devidamente atualizadas, caso contrário o usuário enfrentará problemas na próxima etapa de instalação.


% dism.exe /online /enable-feature /featurename:Microsoft-Windows-Subsystem-Linux /all /norestart --> Comando pra configuração direta pelo cmd!

\paragraph{Instalando o Ubuntu:} Após a reinicialização do sistema, é preciso instalar efetivamente a distribuição Linux desejada, neste caso, o ambiente Ubuntu. Para tanto, abra o aplicativo ``Microsoft Store'' e pesquise por ``Ubuntu'', faça o download e abra o aplicativo Ubuntu que foi instalado em seu computador.

A primeira vez que estiver acessando o ambiente Ubuntu, será pedido que um usuário e uma senha sejam cadastrados, para que o processo de instalação seja efetuado com sucesso. Guarde estas informações, pois elas podem ser importantes para a utilização do Ubuntu como subsistema do Windows.

Agora, com o Windows configurado e o ambiente Ubuntu devidamente instalado, basta retornar ao tópico \ref{WindowsUsers} para a instalação do \SL. 

\paragraph{Acessando dados do Windows no subsistema Ubuntu:} No terminal  do subsistema Ubuntu é possível acessar os arquivos do Windows, dentro da pastar "mnt/"~ localizada na pasta raiz do sistema.

Caso o SuLi esteja na pasta c: no Windows, é possível acessar a pasta pelo comando: cd ../../mnt/c/suli . Observe que neste exemplo a pasta que possui o código do SuLi se chama "suli".


\paragraph{Observação:} Vale frisar que sempre que for necessário realizar uma ação no terminal de comando do computador, o usuário deverá utilizá-lo através do ambiente Ubuntu. \\

A utilização do ambiente Ubuntu como subsistema fica a cargo do próprio usuário, outros subsistemas compatíveis podem oferecer o mesmo resultado (pesquise sobre outras distribuições para subsistemas!).

%\section*{Anexos}
%\addcontentsline{toc}{section}{Anexos}
%\red{(Tópico um pouco mais específico com infos da seção de metodologia numérica)}

\newpage
\bibliographystyle{apacite}
\bibliography{mybib}

\end{document}
